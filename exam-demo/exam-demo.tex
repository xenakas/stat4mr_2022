% arara: xelatex
\documentclass[12pt]{article}

\usepackage{tikz} % картинки в tikz
\usepackage{microtype} % свешивание пунктуации

\usepackage{array} % для столбцов фиксированной ширины

\usepackage{indentfirst} % отступ в первом параграфе

\usepackage{sectsty} % для центрирования названий частей
\allsectionsfont{\centering}

\usepackage{amsmath, amssymb, amsthm} % куча стандартных математических плюшек

\usepackage{amsfonts}

\usepackage{comment}

\usepackage[top=2cm, left=1.2cm, right=1.2cm, bottom=2cm]{geometry} % размер текста на странице

\usepackage{lastpage} % чтобы узнать номер последней страницы

\usepackage{enumitem} % дополнительные плюшки для списков
%  например \begin{enumerate}[resume] позволяет продолжить нумерацию в новом списке
\usepackage{caption}


\usepackage{hyperref} % гиперссылки

\usepackage{multicol} % текст в несколько столбцов


\usepackage{fancyhdr} % весёлые колонтитулы
\pagestyle{fancy}
\lhead{Final demo}
\chead{DON'T PANIC}
\rhead{2022-demo}
\lfoot{}
\cfoot{}
\rfoot{}
\renewcommand{\headrulewidth}{0.4pt}
\renewcommand{\footrulewidth}{0.4pt}

\let\P\relax
\DeclareMathOperator{\P}{\mathbb{P}}
\DeclareMathOperator{\Cov}{Cov}
\DeclareMathOperator{\E}{\mathbb{E}}
\DeclareMathOperator{\Var}{Var}
\DeclareMathOperator{\Corr}{Corr}
\DeclareMathOperator{\plim}{plim}
\DeclareMathOperator{\pCorr}{pCorr}


\newcommand{\cN}{\mathcal{N}}

\usepackage{todonotes} % для вставки в документ заметок о том, что осталось сделать
% \todo{Здесь надо коэффициенты исправить}
% \missingfigure{Здесь будет Последний день Помпеи}
% \listoftodos - печатает все поставленные \todo'шки


% более красивые таблицы
\usepackage{booktabs}
% заповеди из докупентации:
% 1. Не используйте вертикальные линни
% 2. Не используйте двойные линии
% 3. Единицы измерения - в шапку таблицы
% 4. Не сокращайте .1 вместо 0.1
% 5. Повторяющееся значение повторяйте, а не говорите "то же"



\usepackage{fontspec}
\usepackage{polyglossia}

\setmainlanguage{english}
\setotherlanguages{russian}

% download "Linux Libertine" fonts:
% http://www.linuxlibertine.org/index.php?id=91&L=1
\setmainfont{Linux Libertine O} % or Helvetica, Arial, Cambria
% why do we need \newfontfamily:
% http://tex.stackexchange.com/questions/91507/
\newfontfamily{\cyrillicfonttt}{Linux Libertine O}

\AddEnumerateCounter{\asbuk}{\russian@alph}{щ} % для списков с русскими буквами
% \setlist[enumerate, 2]{label=\asbuk*),ref=\asbuk*}


\begin{document}

Rules: two parts, two hours in total, one A4 cheat sheet is allowed. 

\vspace{1cm}

Part 1. Test part, only numerical answers are checked, 6 questions, each question gives 1 point but no more than 4 points in total. 
This part is very predictable :) 

\begin{enumerate}
    \item (bootstrap) I have a sample $X_1$, \ldots, $X_{100}$.
    
    I generate one naive bootstrap sample $X^{*}_{1}$, \ldots, $X^{*}_{100}$. 

    What is the probability that the first observation will be present in the bootstrap sample 2 times or more?

    \item (welch) We have data for an $AB$-experiment $\bar X_a = 10$, $\bar X_b = 12$, 
    $n_a = 20$, $n_b = 30$, $\sum (X_i^a - \bar X_a)^2 = 100$, $\sum (X_i^b - \bar X_b)^2 = 200$.

    Calculate the standard error of $\bar X_a - \bar X_b$ for the Welch test. 
    
    \item (mw test) I have five results of two runners $A$ and $B$ for the 5 km race: 25:12 (A), 26:34 (B), 27:43 (A), 28:12 (A), 29:05 (B).
    
    Calculate Mann-Whitney statistic $U_A$ that tests the null-hypothesis of equal distributions of time. 
    
    (The statistic $U_A$ should positively depend on the ranks of the runner $A$). 

    \item (multiple comparison) I have 100 hypothesis with independent statistics. 
    The null hypothesis for all 100 cases is actually true, but I don't know this. 
    
    I calculate all p-values. 
    If the two lowest p-value are both lower than $0.05$ I wrongly conclude that not all $H_0$ are true. 
    Otherwise I correctly conclude that all $H_0$ are true. 

    What is the probability that I will get the correct conclusion?

    \item (sample size) My target variable is binary and I wish 
    minimal detectable effect equal to $0.01$, probability of I-error  not greater than $0.02$, 
    probability of II-error not greater than $0.10$, control and experimental group of the same size equal to $n$.

    What is minimal value of $n$?


    \item (anova 1+2) Vasiliy loves to eat shaurma. He has three local shaurma dealers. Vasiliy bought 7 shaurmas from each dealer. 
    and measured their weight. He would like to test the hypothesis that mean weight is the same for all dealers. 

    Total sum of squares is 1000, between sum of squares is 500. 

    Calculate the $F$-statistic to test the hypothesis.
        
\end{enumerate}

\newpage
Part 2. Open part, solutions are required, 4 problems, each problem gives 2 points but no more than 6 points in total.
This part is almost unpredictable :)

\begin{enumerate}
 \item Let random variables $Y_1$, \ldots, $Y_n$ be iid uniform $U[0;1]$.
    Consider the naive bootstrap sample $Y_1^*$, \ldots, $Y_n^*$.

    Find $\Var(Y_1^*)$, $\Cov(Y_1^*, Y_2^*)$, $\Var(\bar Y^*)$.

    \item Winnie-the-Pooh simulteneously tests $h$ null hypothesis using independent samples. 
    All the null hypothesis are true but Winnie does not know it. 
    
    \begin{enumerate}
        \item What is the probability that the highest P-value will be greater than $0.95$?
        \item What is the possible range for the probability in point (a) if exactly one null hypothesis is false?
    \end{enumerate}

    \item The correlation matrix of standardized variables $a$, $b$ and $c$ is given by
    \[
    C = \begin{pmatrix}
        1 & 0.2 & 0 \\
         & 1 &  0.2 \\
         & & 1 \\
    \end{pmatrix}
    \]

    Let $p_1$, $p_2$ and $p_3$ be the principal components. 

    \begin{enumerate}
        \item Express $p_1$ in terms of $a$, $b$ and $c$. 
        \item Express $b$ in terms of $p_1$, $p_2$ and $p_3$. 
        \item How would you restore the second observation of variable $b$ if you know that first and second 
        components for the second observation are equal to $-1$ and $2$ respectively?
    \end{enumerate}

    \item Consider the Mann-Whitney test with possible ties. 
    The variables $X_1$, $X_2$, \ldots, $X_{n_x}$ are iid Poisson with rate $\lambda =1$. 
    The variables $Y_1$, $Y_2$, \ldots, $Y_{n_y}$ are iid Poisson with the same rate, independent from $X$ sample. 

    Let $L$ be the number of all pairs $(X_i, Y_j)$ such that $X_i > Y_j$.

    \begin{enumerate}
        \item Find $\E(L)$, $\Var(L)$. 
        \item What is the probability that the ordered sequence of all $X_i$ and $Y_j$ will start with 
        three or more members from $X$-sample?
    \end{enumerate}

\end{enumerate}


\end{document}
