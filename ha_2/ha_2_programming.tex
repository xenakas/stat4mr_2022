% arara: xelatex
\documentclass[12pt]{article}

\usepackage{physics}


\usepackage{tikz} % картинки в tikz
\usepackage{microtype} % свешивание пунктуации

\usepackage{array} % для столбцов фиксированной ширины

\usepackage{indentfirst} % отступ в первом параграфе

\usepackage{sectsty} % для центрирования названий частей
\allsectionsfont{\centering}

\usepackage{amsmath, amsfonts, amssymb} % куча стандартных математических плюшек

\usepackage{comment}

\usepackage[top=2cm, left=1.2cm, right=1.2cm, bottom=2cm]{geometry} % размер текста на странице

\usepackage{lastpage} % чтобы узнать номер последней страницы

\usepackage{enumitem} % дополнительные плюшки для списков
%  например \begin{enumerate}[resume] позволяет продолжить нумерацию в новом списке
\usepackage{caption}

\usepackage{url} % to use \url{link to web}

\usepackage{fancyhdr} % весёлые колонтитулы
\pagestyle{fancy}
\lhead{Statistics 4 Market Research}
\chead{}
\rhead{HA}
\lfoot{2022-2023}
\cfoot{}
\rfoot{\thepage/\pageref{LastPage}}
\renewcommand{\headrulewidth}{0.4pt}
\renewcommand{\footrulewidth}{0.4pt}



\usepackage{todonotes} % для вставки в документ заметок о том, что осталось сделать
% \todo{Здесь надо коэффициенты исправить}
% \missingfigure{Здесь будет Последний день Помпеи}
% \listoftodos - печатает все поставленные \todo'шки


% более красивые таблицы
\usepackage{booktabs}
% заповеди из докупентации:
% 1. Не используйте вертикальные линни
% 2. Не используйте двойные линии
% 3. Единицы измерения - в шапку таблицы
% 4. Не сокращайте .1 вместо 0.1
% 5. Повторяющееся значение повторяйте, а не говорите "то же"



\usepackage{fontspec}
\usepackage{polyglossia}

\setmainlanguage{english}
\setotherlanguages{english}

% download "Linux Libertine" fonts:
% http://www.linuxlibertine.org/index.php?id=91&L=1
\setmainfont{Linux Libertine O} % or Helvetica, Arial, Cambria
% why do we need \newfontfamily:
% http://tex.stackexchange.com/questions/91507/
\newfontfamily{\cyrillicfonttt}{Linux Libertine O}

%\AddEnumerateCounter{\asbuk}{\russian@alph}{щ} % для списков с русскими буквами
%\setlist[enumerate, 2]{label=\asbuk*),ref=\asbuk*}

%% эконометрические сокращения
\DeclareMathOperator{\Cov}{\mathbb{C}ov}
\DeclareMathOperator{\Corr}{\mathbb{C}orr}
\DeclareMathOperator{\Var}{\mathbb{V}ar}

\let\P\relax
\DeclareMathOperator{\P}{\mathbb{P}}
\DeclareMathOperator{\plim}{\mathrm{plim}}

\DeclareMathOperator{\E}{\mathbb{E}}
% \DeclareMathOperator{\tr}{trace}
\DeclareMathOperator{\card}{card}
\DeclareMathOperator{\pCorr}{\mathrm{p}\mathbb{C}\mathrm{orr}}


\newcommand \hb{\hat{\beta}}
\newcommand \hs{\hat{\sigma}}
\newcommand \htheta{\hat{\theta}}
\newcommand \s{\sigma}
\newcommand \hy{\hat{y}}
\newcommand \hY{\hat{Y}}
\newcommand \e{\varepsilon}
\newcommand \he{\hat{\e}}
\newcommand \z{z}
\newcommand \hVar{\widehat{\Var}}
\newcommand \hCorr{\widehat{\Corr}}
\newcommand \hCov{\widehat{\Cov}}
\newcommand \cN{\mathcal{N}}
\newcommand \RR{\mathbb{R}}
\newcommand \NN{\mathbb{N}}
\newcommand{\cF}{\mathcal{F}}
\newcommand{\cH}{\mathcal{H}}


\begin{document}

\section*{Home Assignment 2.}





\begin{enumerate}
	
	\item Set random seed to $M*100 + D$, where $M$ and $D$ is  your Month and Date of birth, respectively.
	
	Generate two samples $X$ of size 20   and $Y$ of size 200 s.t.
	$E(X) \sim Unif[0,1], E(Y) \sim Unif[0,1]$
	from 
	
	\begin{itemize}
		\item two normal distributions with the same variance $\sigma^2_X = \sigma^2_Y = 0.1$;
		\item two normal distributions with 
		if $M$ is odd $\sigma^2_X = 0.1$  and  $\sigma^2_Y = 1$, else
		$\sigma^2_X = 1$  and  $\sigma^2_Y = 0.1$ 
		\item two normal distributions with 3 outliers added afterwards $Out_i \sim Exp(10) * (-1)^{U\{1,2\}}$ ;
		\item two exponential distributions.
	\end{itemize}
	
	
	\begin{enumerate}
		\item Conduct $t$-test, Welch test, Mann-Whitney test 10000 times. 
		\item Calculate nominal type I error rate and power (to calculate type I error fix mean for $Y$ the same as for $X$). 
		\item For each case described above plot how nominal power changes for different values of parameters of the distribution. 
		\item Based on the plots, which test is the most appropriate in each of the cases. Will your answer change if both  sample size are increased by 10.
		
		
	\end{enumerate}

	\item  Use dataset \texttt{SIC33.csv} with the following data:
	
	\begin{itemize}
		\item \textit{output} -- Value added.
		\item  \textit{labor}  -- Labor input.
		\item  \textit{capital} -- Capital stock.
	\end{itemize}
	
	Consider the following model $\ln Y = \beta_0 + \beta_1 \ln (K) + \beta_1 \ln (L)  + u$. 
	Suppose we are interested in estimating the quantity $\gamma =  \beta_1 +\beta_2 $ from the data. 
	Test $H_0 : \gamma = 1$ against $H_a : \gamma \neq 1$ using:
	
	
	\begin{enumerate}
		\item Jackknife estimator;
		\item paired bootstrap;
		\item wild bootstrap (bootstrapping the residuals).
		
	\end{enumerate}
	
	How  the procedure should be  changed in case $u$ is heteroskedastic? 
	
	
	\item  Consider $K$ and $L$  from the same dataset \texttt{SIC33.csv}. 
	
	
	\begin{enumerate}
		\item Illustrate principal  components on a scatter plot. 
		\item Calculate the explained variance by each component,  the weight of $K$ in the 1st component and  the weight of 2nd component in the reconstruction of $L$.
		\item Repeat the analysis for $\ln (K)$ and $\ln (L)$. 
	\end{enumerate}
	
	
	


\end{enumerate}






Deadline: \textbf{2022-12-18, 21:00}. 


\end{document}

